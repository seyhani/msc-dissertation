% !TeX root=../../../main.tex
\chapter{مروری بر کار‌های پیشین}
%\thispagestyle{empty} 
ابزار‌های درستی‌سنجی مشخص می‌کنند که آیا سیستم مطابق انتظار رفتار می‌کند یا خیر و در صورتی که ویژگی مورد انتظار توسط سیستم نقض شود این ابزارها می‌توانند مدرکی برای اثبات این مساله، مثل یک مثال نقض، تولید کنند.
اما چنین مدارکی پاسخی به این سوال که چرا سیستم درست کار نمی‌کند نمی‌دهند. 
در نتیجه برای دست یافتن به درک بهتری از اینکه چرا سیستم مطابق انتظار رفتار نمی‌کند طیف وسیعی از پژوهش‌ها برای  پیدا کردن علت خطا انجام شده‌اند.
استدلال مبتنی بر خلاف واقع که توسط لوئیس در 
\cite{lewis1973counterfactuals}
ارائه شده است مبنای پیدا کردن علت پدیده‌ها در متون فلسفه است.
هالپرن و پرل در 
\cite{hp}
فرمولاسیون ریاضی برای علت واقعی را ارائه کردند که مدلی برای به کارگیری استدلال مبتنی بر خلاف واقع برای پیدا کردن علت در سیستم‌های کامپیوتری را فراهم کرد.
در ادامه پژوهش‌هایی در حوزه درستی‌سنجی که از تعریف هالپرن و پرل برای پیدا کردن علت خطا استفاده کرده‌اند را مورد بررسی قرار می‌دهیم.


\section{علت خطا در مثال نقض}
در
\cite{chockler}
نویسندگان سیستم را به صورت یک سیستم انتقال
\lf{Transition System}
در نظر می‌گیرند که در آن هر حالت یک نگاشت از یک مجموعه‌ی متغیر‌های بولی به مقادیر درست و غلط است.
در این پژوهش با استفاده از تعریف علت واقعی در یک مثال نقض یک ویژگی توصیف شده در
LTL
\lf{Linear Temporal Logic}
یک دوتایی‌ متغیر و حالت به عنوان علت واقعی در نظر گرفته می‌شود.
در همین پژوهش یک الگوریتم تقریبی برای پیدا کردن همه‌ی علت‌ها در یک مثال نقض داده شده ارائه شده است و ابزاری برای نمایش این علت‌ها به صورت گرافیکی به کاربر توسعه داده شده و در ابزار درستی‌سنجی
RuleBase PE
متعلق به
IBM
گنجانده شده است.
\begin{figure}
    \centering
    \includegraphics[width=15cm]{chockler.png}
    \caption{رابط کاربری ابزار
        RuleBase PE
    }
    \label{fig:rulebase}
\end{figure}
سیستمی را در نظر بگیرید که در آن زمانی که تراکنش شروع شود سیگنال 
$\texttt{START}$
و زمانی که خاتمه یابد سیگنال 
$\texttt{END}$
منتشر می‌شود.
مدتی پس از اتمام یک تراکنش سیگنال 
$\texttt{STATUS\_VALID}$
به معنی تایید تراکنش منتشر می‌شود.
فرض کنید نیازمندی سیستم به گونه‌ای است که تراکنش جدید نباید قبل از تایید تراکنش قبلی شروع شود. 
این نیازمندی را می‌توانیم در قالب ویژگی 
$\mr{LTL}$
زیر توصیف کنیم:
\begin{align*}
    \boldsymbol{\mr{G}}((\neg \texttt{START} \wedge \neg \texttt{STATUS\_VALID} \wedge \texttt{END}) 
    \ra [\neg \texttt{START}\ \boldsymbol{\mr{U}}\ \texttt{STATUS\_VALID}])
\end{align*}
تصویر
\ref{fig:rulebase}
رابط کاربری ابزار 
RuleBase PE
را پس از پیدا کردن یک مثال نقض برای این ویژگی نشان می‌دهد.

در این تصویر نقاط قرمز علت‌های واقعی هستند که با الگوریتم تقریبی پیاده‌سازی شده پیدا شده‌اند.
 این پژوهش یکی از کاربردی‌ترین استفاده‌ها از توضیح خطا و پیدا کردن علت خطا را نشان می‌دهد. 
در این پژوهش سعی شده است تا علت خطا در یک مثال نقض پیدا شود و به همین دلیل مقدار متغیر‌ها در حالت‌ها به عنوان علت پیدا می‌شوند در حالی که در پژوهش جاری هدف پیدا کردن علت خطا در کل سیستم است و در واقع ساختار‌های سیستم، مثلا وجود یا عدم وجود روابط تعارض یا فعال‌سازی به عنوان علت خطا پیدا می‌شوند.
اما همانند پژوهش جاری در این پژوهش هم به شکل مستقیم و بدون تغییر از تعریف 
HP
استفاده شده است.

\section{علت خطا در سیستم‌های قابل تنظیم}
سیستم‌های قابل تنظیم
\lf{Configurable}
سیستم‌هایی هستند که با امکان افزودن یا کم‌کردن خصیصه‌
\lf{Feature}
های مختلف با تغییر تنظیمات
\lf{Configuration}
آن‌ها وجود دارد.
رفع‌اشکال در این سیستم‌ها چالش بر انگیز است چون تعداد سیستم‌های ممکن با افزایش تعداد خصیصه‌ها به صورت نمایی زیاد می‌شود.
پیدا کردن علت خطا در چنین سیستم‌هایی کمک می‌کند که توسعه‌دهندگان صرفا برای رفع‌ایراد سیستم صرفا روی خصیصه‌ای تمرکز کنند که به عنوان علت خطا پیدا شده است و علاوه بر این به آن‌ها کمک می‌کند تا روش تنظیم‌مجدد
\lf{Reconfiguraton}
مناسب که منجر به خطا نشود را پیدا کنند.
در 
\cite{assmann2021tactile}
وجود یا عدم وجود خصیصه‌ها در تنظیمات یک سیستم به عنوان متغیر‌ها در نظر گرفته شده‌ است و مطابق با تعریف هالپرن و پرل وجود یا عدم وجود خصیصه‌ها به عنوان علت رخداد رفتار‌های قابل مشاهده در سیستم در نظر گرفته می‌شوند.

\section{علت خطا در پروتکل‌های امنیتی}
در
\cite{actions-cause}
یک مدل برای توصیف برنامه‌های هم‌روند و پیدا کردن علت واقعی رخ دادن خطا در آن‌ها با استفاده از تعریف هالپرن و پرل ارائه شده است.
در این روش مجموعه‌ای از برنامه‌ها که با یکدیگر ارتباط دارند در نظر گرفته می‌شوند که اجرای منجر به خطای آن‌ها به شکل یک لاگ
\lf{Log}
ذخیره شده است.
سپس یک زیرمجموعه از عملیات‌های این برنامه‌ها به عنوان علت خطا در نظر گرفته می‌شود.
این روش برای پیدا کردن علت خطا در پروتکل‌های امنیتی مورد بررسی قرار گرفته است و توانسته است ضعف‌هایی را در زیرساخت صدور گواهی
\lf{Certification}
های جاری بر اساس کلید عمومی
\lf{Public Key}
پیدا کند.
در این پژوهش بر خلاف پژوهش جاری عملیات‌های سیستم به عنوان علت خطا در نظر گرفته می‌شوند. 


\section{چک کردن علیت}
در پژوهش 
\cite{causality-checking}
نویسندگان تعریفی از علت‌ واقعی که الهام گرفته از تعریف 
HP
است ارائه می‌کنند و الگوریتم آن‌ها بر اساس این تعریف در حین اجرای فرآیند وارسی مدل
\lf{Model Checking}
علت‌ها را پیدا کرده و در نتیجه در انتهای وارسی مدل اگر سیستم ویژگی مورد نظر را نقض کرد به جای برگرداندن یک مثال نقض، رویداد‌هایی که علت رخداد خطا بوده‌اند را بر می‌گرداند.
در این پژوهش یک منطق برای توصیف یک دنباله از رویداد عملیات‌های سیستم ارائه شده است و فرمول‌های این منطق به عنوان علت خطا در نظر گرفته می‌شوند. 
این پژوهش هم همانند
\cite{chockler}
سعی بر پیدا کردن همه‌ی علت‌های بروز خطا دارد و علت‌ها عملا دنباله‌هایی از اجرای سیستم هستند. 
تفاوت اصلی این کار با پژوهش جاری در این است که در این پژوهش علت خطا در رفتارهای سیستم جستجو می‌شود در حالی که در پژوهش جاری علت خطا در میان عناصر ساختاری سیستم جستجو می‌شود.
این روش تنها برای ویژگی‌های دسترس‌پذیری ارائه شده است.
در
\cite{Caltais-LTL}
نویسندگان این روش‌ را برای ویژگی‌های دلخواه توصیف شده توسط
LTL
تعمیم دادند.

\section{علت واقعی در خودکاره‌های زمان‌دار}
در
\cite{kolbl2020dynamic}
نویسندگان از تعریف 
HP
برای پیدا کردن علت خطا در خودکاره‌های زمان‌دار
\lf{Timed Automata}
استفاده کرده‌اند.
در درستی‌سنجی خودکاره‌های زمان‌دار یک ابزار وارسی مدل بلا درنگ نقض ویژگی‌ را در قالب یک رد تشخیصی زمان‌دار 
\lf{Timed Diagnostic Trace}
که در واقع یک مثال‌نقض است بر می‌گرداند.
یک 
TDT
در واقع یک دنباله متناوب از انتقال تاخیر
\lf{Transition Delay}
و
انتقال عملیات
\lf{Delay Transition}
ها است که در آن مقدار تاخیر‌ها به صورت سمبلیک مشخص شده‌اند.
هدف این پژوهش پیدا کردن مقادیری یا دامنه‌ای از مقادیر برای این تاخیر‌های سمبلیک است که بروز خطا را اجتناب ناپذیر می‌کنند یا به عبارت دیگر علت واقعی هستند.
در این پژوهش اما به صورت مستقیم از تعریف 
HP
استفاده نشده است و بر اساس آن تعریفی برای علت واقعی نقض ویژگی در یک 
TDT
بیان شده است.

\section{چارچوب علیت بر اساس رد سیستم}
در 
\cite{gossler2013general}
نویسندگان این مساله را مطرح می‌کنند که تعریف ارائه شده توسط هالپرن و پرل ذاتا یک مدل بر اساس منطق گزاره‌ای
\lf{Propositional Logic}
است و به همین دلیل برای درستی‌سنجی پردازه‌ها ایده‌آل نیست.
در این پژوهش یک فرمالیسم و تعریف جدید برای علیت بر اساس تعریف 
HP
ارائه می‌شود که در آن از رد‌
\lf{Trace}
های سیستم به جای متغیر‌ها در مدل 
HP
استفاده می‌شود و امکان ترکیب
\lf{Composition}
چند مدل با یکدیگر را فراهم می‌کند.


\section{استدلال مبتنی بر علیت در 
HML
}
در
\cite{decomposing}
نویسندگان از مفهوم استدلال مبتنی در سیستم‌انتقال برچسب‌دار
\lf{Labeled Transition System}
و 
HML
\lf{Hennesy Milner Loigc} \cite{hml}
استفاده کرده‌اند.
در این پژوهش سیستم با استفاده از یک سیستم انتقال برچسب‌دار مدل می‌شود و رفتار ناامن توسط یک فرمول در قالب
HML
توصیف می‌شود.
سپس یک تعریف جدید که برگرفته شده از تعریف
HP
است با استفاده از این مدل‌ها برای علت واقعی بیان می‌شود.
در این تعریف از مفهومی به نام عدم‌وقوع
\lf{Non-Occurrence}
رویدادها که پیش‌تر در 
\cite{causality-checking}
مطرح شده بود استفاده می‌شود.
شهود کلی مفهوم عدم‌وقوع در علیت این است که در کنار اینکه رخ‌دادن برخی از رویداد‌ها منجر به خطا می‌شود، رخ ندادن رویداد‌ها هم می‌تواند به عنوان علت در نظر گرفته شود.
در تعریف ارائه شده در این پژوهش مجموعه‌ای از محاسبه‌
\lf{Computation}
های سیستم به عنوان علت برقراری یک فرمول 
HML
در سیستم که رفتار نا امن
\lf{Unsafe Behavior}
را توصیف می‌کند تعریف می‌شود.
هر محاسبه شامل یک دنباله از عملیات‌های سیستم در کنار تعدادی عملیات دیگر، که عدم وقوع آن‌ها هم جزئی از علت است، در نظر گرفته می‌شود.
به عبارت دیگر یک محاسبه را می‌توان شامل دو جز در نظر گرفت.
جز اول یک اجرای سیستم است که منجر به خطا می‌شود.
جز دوم مجموعه‌ای از اجراهای سیستم‌ است که منجر به خطا نمی‌شوند و حاصل برگ‌برگ‌ شدن
\lf{Interleaving}
برخی از عملیات‌ها در جز اول این محاسبه هستند.
عملیات‌های برگ‌برگ شده عملیات‌هایی هستند که عدم وقوع آن‌ها به عنوان علت بروز 
خطا در نظر گرفته می‌شود.
در این تعریف علت‌ واقعی به گونه‌ای تعریف شده است که محاسباتی‌ که منجر به فعال شدن فرمول 
HML
در سیستم می‌شوند به عنوان علت در نظر گرفته می‌شوند.
در این تعریف شروطی مشابه با شروط موجود در تعریف 
HP
در نظر گرفته شده است.
در 
\cite{causal-hml}
نویسندگان تعریف خود را بهبود دادند تا تطابق بیشتری با تعریف 
HP
داشته باشد.
علاوه بر این در این پژوهش ثابت شده است که این تعریف از علت در سیستم‌هایی که ارتباط همگام
\lf{Synchronized}
شده دارند قابل ترکیب نیست ولی در حالتی که سیستم‌ها ارتباط همگام نداشته باشند امکان ترکیب یا شکستن آن وجود دارد.
نتایج حاصل از این پژوهش یکی از انگیزه‌های اصلی پژوهش جاری بود برای اینکه با انتخاب یک مدل معنایی یا تعریف علیت متفاوت امکان ترکیب آن برای سیستم‌های همگام شده بررسی شود.
در ادامه به بررسی شباهت‌ها و تفاوت‌های این پژوهش و پژوهش جاری می‌پردازیم
اولا در این پژوهش تعریف جدیدی از علت واقعی ارائه شده است در حالی که در پژوهش جاری مستقیما از تعریف ارائه شده در
\cite{hp}
استفاده شده است.
در پژوهش جاری تمرکز بر پیدا کردن یک علت برای بروز خطا در سیستم است در حالی که در این پژوهش همه‌ی علل خطا مورد بررسی قرار می‌گیرند.
پژوهش جاری علل خطا را در ساختار‌های سیستم جستجو می‌کند در حالی که این پژوهش در میان رفتار‌های سیستم به دنبال علل خطا می گردد.

\section{جمع‌بندی}
همان طور که بررسی شد پژوهش‌های متعددی در زمینه‌ی توضیح خطا ارائه شده است که نشان از اهمیت این مساله در فرآیند درستی‌سنجی و اشکال‌زدایی دارد.
همچنین تعریف 
HP
هم مورد توجه زیادی برای پیدا کردن علت خطا قرار گرفته است.
یکی از مهم‌ترین تمایز‌های پژوهش جاری با پژوهش‌های پیشین در المان‌هایی است که در آن علت خطا پیدا می‌شود. 
همانطور که بررسی شد در تمامی پژوهش‌های پیشین در این زمینه علت خطا در میان رفتارهای سیستم جستجو می‌شود. 
اما در پژوهش جاری رویکردی متفاوت استفاده شده است و علت خطا در میان ساختار‌های سیستم، مثلا هم‌روند بودن یا نبودن پردازه، انجام می‌شود.
مساله‌ی دیگری که باید به آن اشاره شود این است که در پژوهش جاری همانند
\cite{chockler,Chockler_Halpern_Kupferman_2008}
به شکل مستقیم و بدون تغییر از تعریف 
HP
استفاده می‌شود.