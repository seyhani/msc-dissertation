% !TeX root=../main.tex
\chapter{دانش پیش‌زمینه}
%\thispagestyle{empty} 
\section{مقدمه}
در این فصل روش پیدا کردن علت خطا در شبکه‌های نرم‌افزاری توضیح داده می‌شود.
در بخش اول معنای عبارات نت‌کت پویا با استفاده از ساختمان رویداد تعریف می‌شود.
در بخش دوم یک مدل علی برای توصیف ساختمان رویداد مطرح می‌شود.
در نهایت بخش سوم شامل استفاده از این روش‌ها برای توضیح خطا در شبکه‌های نرم‌افزاری با استفاده از چند مثال بیان می‌شود.


% \section{معنای عبارات نت‌کت پویا در قالب ساختمان رویداد}
در این بخش ابتدا شیوه‌ی اعمال چند نوع عملیات برای ترکیب ساختمان‌های رویداد تعریف می‌شود.
سپس با استفاده از این عملیات‌ها معنای عبارات نت‌کت پویا توصیف می‌شود.

\begin{definition}{محدودیت}
    فرض کنید
    $\mr{E} = (E,\#,\vdash)$
    یک ساختمان رویداد باشد.
    به ازای یک مجموعه‌ی
    $A \subseteq E$
    محدودیت
    $\mr{E}$
    به
    $A$
    یک ساختمان رویداد به صورت زیر است:
    \begin{align*}
        \mr{E} \lceil A = (A,\#_A,\vdash_A)
    \end{align*}
    که داشته باشیم:
    \begin{align*}
        X \subseteq Con_A & \iff X \subseteq A \wedge X \in Con                 \\
        X \vdash_A e      & \iff X \subseteq A \wedge e \in A \wedge X \vdash e
    \end{align*}
\end{definition}

\begin{definition}
    فرض کنید
    $\mr{E} = (E,\#,\vdash)$
    یک ساختمان رویداد و
    $a$
    یک رویداد باشد.
    ساختمان رویداد
    $a\mr{E} = (E',\#',\vdash')$
    به گونه‌ای تعریف می‌شود که داشته باشیم:
    \begin{align*}
         & E' = \s{(0,a)} \cup \s{(1,e)|e \in E},                                                                       \\
         & e_0' \#' e_1'  \iff \exists e_0,e_1.e_0' = (1,e_0)
        \ \& \ e_1' = (1,e_1) \ \& \ e_0 \# e_1                                                                         \\
         & X \vdash' e' \iff e' = (0,a) \text{ or } [e' = (1,e_1) \ \& \ (0,a)\in X \ \& \ \s{e|(1,e)\in X} \vdash e_1]
    \end{align*}
\end{definition}

\begin{definition}{ساختمان رویداد برچسب‌دار}
    یک ساختمان رویداد برچسب‌دار یک پنج‌تایی به شکل
    $(E,\#,\vdash,L,l)$
    است که در آن
    $(E,\#,\vdash)$
    یک ساختمان رویداد است،
    $L$
    یک مجموعه از برچسب‌ها
    (فاقد عنصر *)
    و
    $l$
    یک تابع به فرم
    $l: E \ra L$
    است که به هر رویداد یک برچسب اختصاص می‌دهد.
    یک ساختمان رویداد برچسب‌دار را به اختصار به صورت
    $(\mr{E},L,l)$
    نشان می‌دهیم.
\end{definition}

\begin{definition}
    در یک ساختمان رویداد رابطه‌ی
    $\doublevee$
    را به صورت زیر تعریف می‌کنیم:
    \begin{align*}
        e \doublevee e' \iff e \# e' \vee e = e'
    \end{align*}
\end{definition}

\begin{definition}
    فرض کنید
    $(\mr{E},L,l)$
    یک ساختمان رویداد برچسب دار باشد و
    $\alpha$
    یک برچسب باشد.
    $\alpha(\mr{E},L,l)$
    را به صورت یک ساختمان رویداد برچسب دار به شکل
    $(\alpha \mr{E},L',l)$
    تعریف می‌کنیم که در آن
    $L' = \s{\alpha}\cup L$
    و به ازای هر
    $e' \in E'$
    داریم:
    $$
        l'(e') = \begin{cases}
            \alpha & \text{ if } e = (0,\alpha) \\
            l(e)   & \text{ if } e = (1,e)
        \end{cases}
    $$
\end{definition}

\begin{definition}
    فرض کنید
    $\mr{E}_0 = (E_0,\#_0,\vdash_0,L_0,l_0)$
    و
    $\mr{E}_1 = (E_1,\#_1,\vdash_1,L_1,l_1)$
    دو ساختمان رویداد برچسب‌دار باشند.
    مجموعه این دو ساختمان رویداد
    $\mr{E}_0 + \mr{E}_1$
    را به صورت یک ساختمان رویداد برچسب‌دار
    $(E,\#,\vdash,L,l)$
    تعریف می‌کنیم که در آن داشته باشیم:
    \begin{align*}
        E = \s{(0,e)|e \in E_0} \cup \s{(1,e)|e \in E_1}
    \end{align*}
    با استفاده از این مجموعه از رویداد‌ها توابع
    $iota_k: E_k \ra E$
    را به صورت زیر و به ازای
    $k=0,1$
    تعریف می‌کنیم:
    \begin{align*}
        \iota_k(e) = (k,e)
    \end{align*}
    رابطه‌ی تعارض را به گونه‌ای تعریف می‌کنیم که داشته باشیم:
    \begin{align*}
        e \# e' \iff & \exists e_0,e_0'. e = \iota_0(e_0)
        \wedge e' = \iota_0(e_0') \wedge e_0 \#_0e_0'             \\
                     & \bigvee \exists e_1,e_1'. e = \iota_1(e_1)
        \wedge e' = \iota_1(e_1') \wedge e_1 \#_1 e_1'            \\
                     & \bigvee \exists e_0,e_1.(e=\iota_1(e_0)
        \wedge e' =\iota_1(e_1)) \vee
        (e'=\iota_1(e_0) \wedge e =\iota_1(e_1))
    \end{align*}
    رابطه‌ی فعال‌سازی را به گونه‌ای تعریف می‌کنیم که داشته باشیم:
    \begin{align*}
        X \vdash e \iff & X \in Con \wedge e \in E \wedge                   & \\
                        & (\exists X_0 \in Con_0,e_0 \in E_0.X = \iota_0X_0
        \wedge e = \iota_0(e_0) \wedge X_0 \vdash_0 e_0) \text{ or }          \\
                        & (\exists X_1 \in Con_1,e_1 \in E_1.X = \iota_1X_1
        \wedge e = \iota_1(e_1) \wedge X_1 \vdash_1 e_1)                      \\
    \end{align*}
    و مجموعه‌ی برچسب‌ها را به صورت
    $L = L_0 \cup L_1$
    و تابع برچسب‌گذاری را به صورت زیر تعریف می‌کنیم:
    $$
        l(e) = \begin{cases}
            l_0(e_0) & \text{ if } e = \iota_0(e_0) \\
            l_1(e_1) & \text{ if } e = \iota_1(e_1)
        \end{cases}
    $$
\end{definition}

\begin{definition}
    فرض کنید که
    $\mr{E_0} = (E_0,\#_0,\vdash_0,L_0,l_0)$
    و
    $\mr{E_1} = (E_1,\#_1,\vdash_1,L_1,l_1)$
    دو ساختار رویداد برچسب‌گذاری شده باشند.
    حاصلضرب آن‌ها
    $\mr{E}_0 \times \mr{E}_1$
    را به صورت یک ساختمان رویداد برچسب‌گذاری شده
    $\mr{E} = (E,\#,\vdash,L,l)$
    تعریف می‌کنیم که در‌ آن رویداد‌ها به صورت زیر تعریف می‌شوند:
    \begin{align*}
        E_0 \times_* E_1 =
        \s{(e_0,*)|e_0 \in E_0}
        \cup \s{(*,e_1)|e_1 \in E_1}
        \cup \s{(e_0,e_1)| e_0 \in E_0 \wedge e_1 \in E_1}
    \end{align*}
    با توجه به این مجموعه‌ رویداد‌ها توابعی به شکل
    $\pi_i: E \ra_* E_i$
    تعریف می کنیم که به ازای
    $i=0,1$
    داشته باشیم:
    $\pi_i(e_0,e_1) = e_i$.
    در اینجا رابطه‌ی تعارض را به کمک رابطه‌ی
    $\doublevee$
    که پیش‌تر تعریف شد، به شکل زیر به ازای تمامی رویداد‌های
    $e,e' \in E$
    توصیف می‌کنیم:
    \begin{align*}
        e \doublevee e' \iff
        \pi_0(e)\doublevee_0 \pi_0(e')
        \vee \pi_1(e)\doublevee_1\pi_1(e')
    \end{align*}
    رابطه‌ی فعال‌سازی  را به صورت زیر تعریف می‌کنیم:
    \begin{align*}
         & X \vdash e \iff X \in Con \wedge e \in \mathcal{E} \wedge        \\
         & (\pi_0(e)\text{ is defined } \Rightarrow \pi_0X\vdash_0\pi_0(e))
        \wedge (\pi_1(e)\text{ is defined } \Rightarrow \pi_1X\vdash_1\pi_1(e))
    \end{align*}
    مجموعه‌ی برچسب‌های حاصلضرب را به صورت زیر تعریف می‌کنیم:
    \begin{align*}
        L_0 \times_* L_1 = \s{ (\alpha_0,*)|\alpha_0 \in L_0}
        \cup \s{(*,\alpha_1)|\alpha_1 \in L_1}
        \cup \s{(\alpha_0,\alpha_1)|\alpha_0 \in L_0 \wedge \alpha_1 \in L_1}
    \end{align*}
    در انتها تابع برچسب‌گذاری را به صورت زیر تعریف می‌کنیم:
    \begin{align*}
        l(e) = (l_0(\pi_0(e),l_1(\pi_1(e))))
    \end{align*}
\end{definition}

\begin{definition}
    فرض کنید که 
    $\mr{E} = (E,\#,\vdash,L,l)$
    یک ساختمان رویداد برچسب‌دار باشد.
    فرض کنید
    $\Lambda$
    یک زیرمجموعه از 
    $L$
    باشد.
    محدودیت 
    $\mr{E}$
    به 
    $\Lambda$
    را به صورت 
    $\mr{E}\Lambda$
    و به شکل یک ساختمان رویداد برچسب‌گذاری شده به شکل 
    $(E',\#',\vdash',L\cap\Lambda,l')$ 
    که در آن
    $(E',\#',\vdash') = (E,\#,\vdash)\lceil \s{e \in E|l(e)\in \Lambda}$
    است و تابع برچسب‌گذاری معادل محدودسازی تابع
    $l$
    به دامنه‌ی 
    $L\cap \Lambda$
    است.
\end{definition}

در ادامه چگونگی تعریف معنا برای عبارت‌های نت‌کت پویا بیان می‌شود.
در نت‌کت پویا رفتار عبارت‌های نت‌کت فقط به صورت انتها به انتها در نظر گرفته می‌شوند.
با توجه به همین موضوع در ادامه قسمتی از نت‌کت پویا مورد استفاده قرار می‌گیرد که عبارت‌های نت‌کت به صورت نرمال در آن ظاهر می‌شوند.
اگر فرض کنیم که مجموعه‌ی فیلد‌ها 
$f_1,f_2,...,f_k$
باشد یک فیلتر کامل به صورت 
$\alpha = f_1 = n_1 ... f_k = n_k$
و یک اختصاص کامل به صورت 
$\pi = f_1 \la n_1 ... f_k \la n_k$
تعریف می‌شود.
می‌گوییم یک عبارت 
$q$
در
$NetKAT^{-dup}$
به فرم نرمال است 
اگر به شکل
$\sigma_{\alpha\cdot\pi \in \mathcal{A}}\alpha\cdot\pi$
باشد که داشته باشیم
$\mathcal{A} = \s{\alpha_i\cdot\pi_i | i \in I}$.
در عبارت قبل 
$I$ 
مدل زبانی
$NetKAT^{-dup}$
می‌باشد.
بر اساس لم ۵ در 
به ازای هر عبارت 
$p$
در
$NetKAT^{-dup}$
یک عبارت 
$p'$
به فرم نرمال وجود دارد که داشته باشیم:
$p\equiv p'$

در ادامه از گرامر زیر برای توصیف عبارات شبکه استفاده می‌کنیم:
\begin{align*}
    F ::= & \alpha\cdot\pi \\
    D ::= & \bot | F;D | x?F;D | x!F;D | D \parallel D | D \oplus D
\end{align*}
بیان گرامر بالا همچنان به اندازه‌ی نت‌کت پویا است.
در ادامه فرض کنیم که 
$\mathcal{A}$
مجموعه‌ی الفبا شامل تمامی حروف به شکل
$\alpha\cdot\pi,x?F,x!F$
باشد و داشته باشیم
$\alpha \in \mathcal{A}$.
معنای عبارت‌های بر روی زبان را به صورت زیر تعریف می‌کنیم:
\begin{align*}
    \sem{\bot} & = (\e,\e) \\
    \sem{\alpha;t} & = \alpha(\sem{t}) \\
    \sem{t_1 \oplus t_2} & = \sem{t_1} + \sem{t_2} \\
    \sem{t_1 \parallel t_2} & = \sem{t_1} \times \sem{t_2} \\
    \sem{\delta_{L}(t)} & = \sem{t}\lceil \mathcal{A} \setminus L
\end{align*}

\section{مدل علی برای ساختمان رویداد}
در ادامه نحوه‌ی توصیف یک ساختمان رویداد در قالب یک مدل علی را بیان می کنیم.

فرض کنیم که
$\mr{E} = (E,\#,\vdash)$
یک ساختمان رویداد باشد.
مدل علی این ساختمان رویداد را به صورت
$\mc{M} = (\mc{s},\mc{F},\mc{E})$
تعریف می‌کنیم که در آن
$\mc{S} = (\mc{U},\mc{V},\mc{R})$.
در این مدل فرض می‌کنیم همه‌ی متغیر‌ها از نوع بولی هستند.
همچنین در این مدل متغیر برونی در نظر نمی‌گیریم بنابراین داریم
$\mc{U} = \e$.
اگر فرض کنیم مجموعه‌ رویدادها به صورت
$E = \s{e_1,e_2,...,e_n}$
باشد مجموعه‌ی متغیر‌های درونی را به صورت زیر تعریف می‌کنیم:
\begin{align*}
    \mathcal{V} = & \s{C_{e_i,e_j} |  1 \leq i < j \leq n.
    e_i \in E \wedge e_j \in E}                              \\
                  & \cup \s{EN_{s,e} | s \in \mathcal{P}(E),
    e \in E. e \not \in s }                                  \\
                  & \cup \s{M_{s,e} | s \in \mathcal{P}(E),
        e \in E. e \not \in s } \cup \s{PV}
\end{align*}
به صورت شهودی به ازای هر عضو از رابطه‌های
$\#,\vdash,\vdash_{min}$
یک متغیر درونی در نظر می‌گیریم که درست بودن این متغیر به معنای وجود عنصر منتاظر با آن در رابطه است.
به ازای
$x,y \in \mc{P}(E)$
پوشیده شدن 
$x$
توسط
$y$
را که با 
$x \prec y$
نمایش می‌دهیم به صورت زیر تعریف می‌کنیم:
\begin{align*}
    x \subseteq y \wedge x \neq y \wedge
    (\forall z. x \subseteq z \subseteq y \Rightarrow x = z
    \text{ or } y = z)
\end{align*}
همچنین به ازای هر متغیر 
$X \in \mc{V}$
بردار
$\vec V_X$
را بردار شامل همه‌ی متغیر‌های درونی به غیر از 
$X$
تعریف می‌کنیم.
با استفاده از این تعاریف 
توابع متغیر‌های درونی را به صورت زیر تعریف می‌کنیم:
$$
    \f{C_{e,e'}} = \begin{cases}
        true  & \text{ if } e \# e'\\
        false & \text{ otherwise }
    \end{cases}
$$
$$
    \f{M_{s,e}} = \begin{cases}
        Min(s,e) \wedge Con(s) & \text{ if } s \vdash_{min} e \\
        false                  & \text{ otherwise }
    \end{cases}
$$
\begin{align*}
    \f{EN_{s,e}} & =
    \left(
    M_{s,e} \vee
    \left(
    \bigvee_{s'\prec s}EN_{s',e}
    \right)
    \right)
    \bigwedge
    Con(s)
\end{align*}
که در آن‌ها داریم:
\begin{align*}
    Con(s)   & =   \left(
    \bigwedge_{ 1\leq j<j' \leq n \wedge e_j,e_{j'} \in s}
    \neg C_{e_j,e_{j'}}
    \right)               \\
    Min(s,e) & = \left(
    \bigwedge_{s' \subseteq E. (s' \subset s \vee s \subset s')
        \wedge e \notin s'}
    \neg M_{s',e}
    \right)
\end{align*}
فرض کنید که
$\mathbb{E}$
مجموعه‌ی تمامی سه‌تایی‌ها به فرم
$(E,\#',\vdash')$
باشد که داشته باشیم:
\begin{align*}
    \#' \subseteq E \times E \\
    \vdash' \subseteq \mc{P}\times E
\end{align*}
یک تابع به فرم
$ES: \times_{V \in \mathcal{V}\setminus \s{PV}} \mathcal{R}(V) \rightarrow \mathbb{E}$
تعریف می‌کنیم که به صورت شهودی ساختمان رویداد حاصل از مقدار فعلی متغیر‌ها در مدل علی را به دست می‌دهد.
فرض کنیم 
$\vec v$
برداری شامل مقادیر متغیرهای
$\mc{V} \setminus \s{PV}$
باشد.
به ازای هر متغیر مانند 
$V \in \mc{V}$
مقدار آن در 
$\vec v$
را با
$\vec v(V)$
نمایش می‌دهیم.
تابع 
$ES$
را به گونه‌ای تعریف می‌کنیم که اگر 
$ES(\vec v) = (E,\#',\vdash')$
آنگاه داشته باشیم:
\begin{align*}
    \forall e,e' \in E. e \#' e' \wedge e' \#' e
     & \iff \vec{v}(C_{e,e'}) = \T \\
    \forall s \in \mathcal{P}(E), e \in E.  s \vdash' e
     & \iff \vec{v}(EN_{s,e}) = \T
\end{align*}
در نهایت فرض می‌کنیم که رفتار بد سیستمی که در قالب ساختمان رویداد مدل شده است، در قالب تابع متغیر 
$PV$
توصیف شده است و در صورتی که رفتار بد در سیستم وجود داشته باشد مقدار آن درست و در غیر این صورت غلط است.
با استفاده از مدل علی که به این شکل توصیف شود برای پیدا کردن علت خطا کافی است علت واقعی 
$PV=\T$
را در مدل علی و مطابق تعریف پیدا کنیم.


