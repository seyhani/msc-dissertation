% !TeX root=../../../main.tex
\chapter{جمع‌بندی و کار‌های آینده}
%\thispagestyle{empty} 
\section{جمع‌بندی کار‌های انجام‌شده}
 در این پژوهش روشی برای استفاده از تعریف علت واقعی مطابق
\cite{hp}
برای پیدا کردن علت خطا در برنامه‌های توصیف شده در زبان نت‌کت پویا ارائه شد.

در این پژوهش سعی شد تا از مفهوم علت واقعی برای پیدا کردن علت خطا در شبکه‌های نرم‌افزاری استفاده شود.
زبان نت‌کت پویا برای توصیف شبکه‌های نرم‌افزاری انتخاب شد.
برای این منظور زبان نت‌کت پویا به عنوان زبان برنامه‌ی‌ شبکه‌ انتخاب شد.
دلیل انتخاب نت‌کت پویا این است که اولا این زبان چون بر اساس نت‌کت بنا شده سادگی و مینیمال بودن خود را حفظ کرده است که این مساله به ساده‌تر شدن مساله کمک می‌کند.
ثانیا نت‌کت پویا امکان به روز‌ رسانی شبکه را فراهم می‌کند و از آن‌جایی که هدف نهایی پیدا کردن علت خطا تسهیل کردن فرآیند رفع خطا است، وجود ساختارهایی در زبان برای به روز رسانی‌های شبکه کمک می کند تا این ساختار ها هم در پیدا کردن علت نقش داشته باشند و کمک بیشتری به رفع خطا کنند.
 برای اختصاص معنا به عبارات نت‌کت پویا از ساختمان رویداد استفاده شده است.
 در نهایت برای بر اساس تعریف مدل علّی ارائه شده در
 \cite{hp}
 مدل علّی ساختمان رویداد در این پژوهش ارائه شده است.
 برای بررسی کارایی این مدل چند دسته از ویژگی‌های مطرح شبکه مورد بررسی قرار گرفته‌اند و تطابق علت واقعی پیدا شده با شهود موجود از مساله مورد بحث قرار گرفته است.

 \section{نوآوری‌ها و دستاورد‌ها}

 \subsection{جستجو در ساختار}
 \subsection{ساختمان رویداد}
 \subsection{استفاده مستقیم از تعریف علت}
در این پژوهش سعی شد تا از رویکرد متفاوتی نسبت به روش‌های پیشین برای توصیف خطا استفاده شود.
در پژوهش‌هایی مانند
\cite{causality-checking,causal-hml,chockler}
المان‌هایی در رفتار یا وضعیت سیستم به عنوان علت بروز خطا در نظر گرفته می‌شوند.
اما در این پژوهش در میان المان‌های ساختاری سیستم به جستجوی علت پرداخته شده است.
یکی از مزایایی که برای این روش می‌توان در نظر گرفت استفاده آسان‌تر از آن برای فرآیند تعمیر یا سنتز خودکار نرم‌افزار است. 
چون در اینجا ساختار‌های سیستم به عنوان علت پیدا می‌شوند این موضوع به تولید تعمیر خودکار کمک می‌کند.
مساله‌ی دیگر استفاده از ساختمان رویداد به عنوان مدل معنایی است.
همانطور که پیش‌تر بیان شد ساختمان رویداد یک مدل غیرجایگذاری شده است که در آن هم‌روندی به صورت صریح مشخص می‌شود.
این موضوع سبب می‌شود که اولا در فرآیند پیدا کردن علت، ساختار‌های جعلّی مانند جایگذاری پردازه‌های هم روند که یک ترتیب برای اجرای آن‌ها ایجاد می‌کند به عنوان علت پیدا نشوند، ثانیا بتوان هم‌روندی دو پردازه یا عملیات را به عنوان علت تعریف کرد، کاری که در مدل‌های جایگذاری شده به دلیل اینکه هم‌روندی صراحتا توصیف نمی‌شود ممکن نیست.
یکی دیگر از تفاوت‌های این روش با روش‌هایی مانند
\cite{causal-hml,causality-checking}
در استفاده از تعریف علت واقعی است.
در پژوهش‌های ذکر شده با اقتباس از تعریف
HP
تعریف جدیدی برای علت واقعی ارائه شده است و در مورد معادل بودن این تعاریف بحثی نشده است.
در مقابل در پژوهش جاری به صورت مستقیم از تعریف 
HP
استفاده شده است.

\section{محدودیت‌ها}
\subsection{پیچیدگی زمانی}
یک ساختمان رویداد با 
$n$
رویداد را در نظر بگیرید.
مدل علّی این ساختمان رویداد شامل 
$O(n2^n)$
متغیر است.
برای پیدا کردن علت واقعی در این مدل و به طور خاص برای بررسی شرط ۲.ب لازم است تا تمامی زیر مجموعه‌های یک افراز از این متغیر‌ها بررسی شود که در بهترین حالت پیچیدگی زمانی
$O(2^{n2^n})$
دارد.
بنابراین پیاده‌سازی این روش بدون بهینه‌سازی یا استفاده از روش‌های ابتکاری عملا ممکن نیست.
\subsection{توصیف خطا در سطح مدل علی}
در روش ارائه شده در این پژوهش برای پیدا کردن علت خطا در یک برنامه توصیف شده در زبان نت‌کت پویا، لازم است تا رفتار نا امن در قالب یک تابع در مدل علّی و به عنوان یک شرط بر روی مجموعه‌ی پیکربندی‌های ساختمان رویداد منتج شده از‌ آن توصیف شود.
این مساله کار کردن با این روش را برای کاربر سخت می‌کند. 
راه حل مناسب ارائه یک منطق در سطح زبان است که به کاربر اجازه‌ی توصیف رفتار نا امن یا در روش بهتر اجازه‌ی توصیف ویژگی مورد نظر در قالب یک منطق را بدهد.

\subsection{استدلال در مورد یک علت}
روش ارائه شده در این پژوهش می‌تواند برای اثبات اینکه چه ساختاری از برنامه علت خطا است به کار رود. 
ولی این مساله به تنهایی برای تسهیل فرآیند اشکال‌زدایی سیستم کافی نیست.
برای اینکه علت خطا بتواند به شکل کاربردی در فرآیند اشکال‌زدایی مورد استفاده قرار گیرد لازم است تا مشابه روش‌هایی مانند
\cite{chockler}
تمامی علت‌های ممکن برای خطا پیدا شوند و به کاربر نمایش داده شوند.

\section{کار‌های آینده}

\subsection{ترکیب علت}
در 
\cite{causal-hml}
نویسندگان ثابت کرده‌اند که امکان ترکیب علت‌ها در اجزای یک پردازه برای پیدا کردن علت در آن پردازه در شرایطی که پردازه‌ها ارتباط 


\subsection{سنتز تعمیر}
از علت‌های پیدا شده در سیستم برای بروز خطا می‌توان برای سنتز تعمیر مناسب برای سیستم استفاده کرد.
به طور خاص در این روش که علت خطا در ساختار‌های سیستم جستجو می‌شود سنتز کردن تعمیر فرآیند ساده‌تری خواهد بود.
