% !TeX root=../../../main.tex
\chapter{جمع‌بندی و کار‌های آینده}
%\thispagestyle{empty} 
\section{جمع‌بندی کار‌های انجام‌شده}
 در این پژوهش روشی برای استفاده از تعریف علت واقعی مطابق
\cite{hp}
برای پیدا کردن علت خطا در برنامه‌های توصیف شده در زبان نت‌کت پویا ارائه شد.

برای امکان پذیر شدن استفاده از مدل علّی هالپرن و پرل از ساختمان رویداد به عنوان مدل معنایی برنامه‌های نت‌کت پویا استفاده شد.
سپس یک مدل علّی برای توصیف ساختمان رویداد در قالب معادلات ساختاری مطابق
\cite{hp}
بیان شد که در آن رفتار نا امن شبکه در قالب معادله‌ی یکی از متغیر‌ها توصیف شده است.
به صورت شهودی مدل علّی ساختمان رویداد این امکان را فراهم می‌کند که بتوان وجود رفتار نا امن را در ساختمان رویداد‌هایی که ناشی از اعمال تغییر در المان‌های ساختاری آن هستند را بررسی کرد، مثلا افزودن یک رابطه‌ی تعارض یا حذف یک رابطه‌ی فعال‌سازی.
استفاده از این مدل معنایی همچنین این امکان را فراهم کرد که بتوان از تعریف علت واقعی ارائه شده توسط هالپرن و پرل به شکل مستقیم و بدون تغییر استفاده کرد.
در نهایت با استفاده از روش ارائه شده چند نمونه از ویژگی‌های رایج شبکه مورد تحلیل قرار گرفتند تا علت واقعی نقض ویژگی در آن‌ها پیدا شود.
همانطور که پیش‌تر در 
\cite{hp}
به آن اشاره شده، معیار مشخصی برای بررسی کیفیت یک فرمولاسیون علت واقعی وجود ندارد.  
به همین دلیل در این پژوهش هم صرفا میزان تطابق علت‌های پیدا شده با شهود موجود از مساله بررسی شد.  

 \section{نوآوری‌ها و دستاورد‌ها}

 \subsection{جستجو در ساختار}
همانطور که پیش‌تر بیان شد یکی از تفاوت‌های اصلی این پژوهش با پژوهش‌های پیشین مانند
\cite{causality-checking,chockler,causal-hml}
در زمینه‌ی توضیح خطا، امکان معرفی ساختار‌های سیستم، مثلا هم‌روند بودن دو عملیات، به عنوان علت خطا بوده است.
اما در پژوهش‌های پیشین رفتار‌های سیستم، مثلا یک دنباله از عملیات‌های سیستم، به عنوان علت خطا در نظر گرفته می‌شدند.


 \subsection{استفاده از ساختمان رویداد به عنوان مدل معنایی}
در این پژوهش به جای استفاده از سیستم انتقال برای مدل معنایی برنامه‌های نت‌کت پویا از ساختمان رویداد استفاده شد.
ساختمان رویداد همانند سیستم انتقال یک مدل محاسباتی برای پردازه‌های هم‌روند است. 
اما بر خلاف سیستم انتقال که یک مدل برگ‌برگ شده است، ساختمان رویداد یک مدل غیر برگ‌برگ شده است.
در مدل‌های برگ‌برگ‌ شده، هم‌روندی پردازه‌ها با انتخاب غیرقطعی میان ترتیب‌های مختلف اجرای آن‌ها توصیف می‌شود.
به عنوان مثال دو پردازه‌ی 
$a$
و
$b$
را در نظر بگیرید که به ترتیب یک عملیات 
$a$
و
$b$
انجام داده و متوقف می‌شوند.
در مدل‌های برگ‌برگ‌ شده رفتار 
$a\parallel b$
با رفتار 
$ab + ba$
معادل است، چون هم‌روندی پردازه‌ها صراحتا در مدل ذکر نمی‌شود.
در طرف مقابل، در ساختمان رویداد این هم‌روندی به شکل صریح توصیف می‌شود.
مثلا در مثال بالا نبود رابطه‌ی تعارض و فعال‌سازی بین رویداد‌های 
$a$
و
$b$
به معنای هم‌روندی آن‌ها است که صراحتا در مدل قید شده است.
یک از مزیت‌های استفاده از ساختمان رویداد در پژوهش جاری این است که امکان تعریف هم‌روندی دو عملیات به عنوان علت واقعی را فراهم می‌کند.


 \subsection{استفاده مستقیم از تعریف علت}
بر خلاف کار‌های پیشین مانند
\cite{decomposing,causality-checking,Caltais-LTL,causal-hml}
که در آن‌ها از تعریف جدیدی از علت واقعی بر اساس تعریف 
HP
استفاده شده است، در پژوهش جاری سعی شد تا مستقیما و بدون تغییر از تعریف علت واقعی ارائه شده در 
\cite{hp}
استفاده شود.
مزیت این رویکرد نسبت به استفاده از یک تعریف جدید این موضوع است که هنوز معیار مشخصی برای مقایسه‌ی تعاریف علت واقعی وجود ندارد و به همین دلیل امکان ارزیابی تعریف جدید وجود ندارد.
در پژوهش جاری با استفاده از تعریف یک مدل علّی در قالب معادلات ساختاری این امکان فراهم شد تا تعریف 
HP
مستقیما مورد استفاده قرار گیرد.

\section{محدودیت‌ها}
\subsection{پیچیدگی زمانی}
یک ساختمان رویداد با 
$n$
رویداد را در نظر بگیرید.
مدل علّی این ساختمان رویداد شامل 
$O(n2^n)$
متغیر است.
برای پیدا کردن علت واقعی در این مدل و به طور خاص برای بررسی شرط ۲.ب لازم است تا تمامی زیر مجموعه‌های یک افراز از این متغیر‌ها بررسی شود که در بهترین حالت پیچیدگی زمانی
$O(2^{n2^n})$
دارد.
بنابراین پیاده‌سازی این روش بدون بهینه‌سازی یا استفاده از روش‌های ابتکاری عملا ممکن نیست.
\subsection{توصیف خطا در سطح مدل علی}
در روش ارائه شده در این پژوهش برای پیدا کردن علت خطا در یک برنامه توصیف شده در زبان نت‌کت پویا، لازم است تا رفتار نا امن در قالب یک تابع در مدل علّی و به عنوان یک شرط بر روی مجموعه‌ی پیکربندی‌های ساختمان رویداد منتج شده از‌ آن توصیف شود.
این مساله کار کردن با این روش را برای کاربر سخت می‌کند. 
راه حل مناسب ارائه یک منطق در سطح زبان است که به کاربر اجازه‌ی توصیف رفتار نا امن یا در روش بهتر اجازه‌ی توصیف ویژگی مورد نظر در قالب یک منطق را بدهد.

\subsection{استدلال در مورد یک علت}
روش ارائه شده در این پژوهش می‌تواند برای اثبات اینکه چه ساختاری از برنامه علت خطا است به کار رود. 
ولی این مساله به تنهایی برای تسهیل فرآیند اشکال‌زدایی سیستم کافی نیست.
برای اینکه علت خطا بتواند به شکل کاربردی در فرآیند اشکال‌زدایی مورد استفاده قرار گیرد لازم است تا مشابه روش‌هایی مانند
\cite{chockler}
تمامی علت‌های ممکن برای خطا پیدا شوند و به کاربر نمایش داده شوند.

\section{کار‌های آینده}

\subsection{ترکیب علت}
در 
\cite{causal-hml}
نویسندگان ثابت کرده‌اند که امکان ترکیب علت‌ها در اجزای یک پردازه برای پیدا کردن علت در آن پردازه در شرایطی که پردازه‌ها ارتباط  
همگام دارند وجود ندارد.
قدم بعدی این پژوهش اثبات امکان ترکیب علت‌ها برای پیدا کردن علت در یک پردازه‌ی بزرگ‌تر است.
این مساله اولا تفسیر علت به دست آمده را ساده‌تر می‌کند ثانیا می‌تواند باعث کاهش پیچیدگی زمانی پیدا کردن علت در یک پردازه‌ی مرکب شود.

\subsection{سنتز تعمیر}
با توجه به اینکه علت‌های پیدا شده در این پژوهش المان‌های ساختاری سیستم هستند، مثلا وجود هم‌روندی میان دو عملیات، عملا این علت چگونگی رفع این مشکل در سیستم‌ را نشان می‌دهد. مثلا اگر وجود هم‌روندی علت به وجود آمد خطا در یک سیستم باشد برای رفع آن می‌توان یک ترتیب میان دو عملیات ایجاد کرد.
اگر چگونگی پیاده‌سازی این ترتیب در سطح زبان نت‌کت پویا مشخص شود عملا می‌توان برای از علت خطا برای سنتز خودکار تعمیر برنامه استفاده کرد.

\subsection{مقایسه و رتبه‌بندی علت‌ها}
هالپرن و پرل در 
\cite{hp2}
مفهوم مسئولیت%
\lf{Responsibility}
را در مدل علّی خود تعریف کرده‌اند.
این مفهوم کمک می‌کند تا بتوان میان علت‌ها تمایز قائل شد و یک معیار کمی به دست می‌دهد که بتوان علت‌ها را با یکدیگر مقایسه کرد.
به عنوان مثال یک انتخابات را در نظر بگیرید که در آن دو کاندیدا وجود دارد و کسی که اکثریت آرا از میان ۱۱ رای را کسب کند برنده انتخابات است. 
دو سناریو را در نظر بگیرید که در اولی نفر برنده با آرای ۶ به ۵ و در سناریوی دوم با آرای ۱۰ به ۱ برنده انتخابات می‌شود.
واضح است که رای هر نفر به فرد برنده در سناریوی اول اهمیت بیشتری نسبت به سناریوی دوم دارد چون برگرداندن رای هر نفر در سناریوی اول می‌تواند نتیجه‌ی انتخابات را تغییر دهد.
در
\cite{hp2}
مفهوم مسئولیت به گونه‌ای تعریف شده است که به رای هر فرد به نفر برنده در سناریوی‌ اول مسئولیت بیشتری، که یک مقدار عددی است، 
اختصاص می‌دهد.
برای کمک گرفتن از میزان مسئولیت در این پژوهش می‌توان پس از پیدا کردن چندین علت مختلف برای بروز خطا در یک شبکه، آن‌ها را بر اساس میزان مسئولیت شان مرتب کرد و سپس به کاربر نمایش داد تا کاربر راحت‌تر بتواند علت‌های مهم‌تر را شناسایی کند و از آن‌ها بهره ببرد.