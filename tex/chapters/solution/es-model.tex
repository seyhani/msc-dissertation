\section{مدل علی برای ساختمان رویداد}
در ادامه نحوه‌ی توصیف یک ساختمان رویداد در قالب یک مدل علی را بیان می کنیم.

فرض کنیم که
$\mr{E} = (E,\#,\vdash)$
یک ساختمان رویداد باشد.
مدل علی این ساختمان رویداد را به صورت
$\mc{M} = (\mc{s},\mc{F},\mc{E})$
تعریف می‌کنیم که در آن
$\mc{S} = (\mc{U},\mc{V},\mc{R})$.
در این مدل فرض می‌کنیم همه‌ی متغیر‌ها از نوع بولی هستند.
همچنین در این مدل متغیر برونی در نظر نمی‌گیریم بنابراین داریم
$\mc{U} = \e$.
اگر فرض کنیم مجموعه‌ رویدادها به صورت
$E = \s{e_1,e_2,...,e_n}$
باشد مجموعه‌ی متغیر‌های درونی را به صورت زیر تعریف می‌کنیم:
\begin{align*}
    \mathcal{V} = & \s{C_{e_i,e_j} |  1 \leq i < j \leq n.
    e_i \in E \wedge e_j \in E}                              \\
                  & \cup \s{EN_{s,e} | s \in \mathcal{P}(E),
    e \in E. e \not \in s }                                  \\
                  & \cup \s{M_{s,e} | s \in \mathcal{P}(E),
        e \in E. e \not \in s } \cup \s{PV}
\end{align*}
به صورت شهودی به ازای هر عضو از رابطه‌های
$\#,\vdash,\vdash_{min}$
یک متغیر درونی در نظر می‌گیریم که درست بودن این متغیر به معنای وجود عنصر منتاظر با آن در رابطه است.

به ازای
$x,y \in \mc{P}(E)$
پوشیده
شدن 
\lf{Covering}
$x$
توسط
$y$
را که با 
$x \prec y$
نمایش می‌دهیم به صورت زیر تعریف می‌کنیم:
\begin{align*}
    x \subseteq y \wedge x \neq y \wedge
    (\forall z. x \subseteq z \subseteq y \Rightarrow x = z
    \text{ or } y = z)
\end{align*}
همچنین به ازای هر متغیر 
$X \in \mc{V}$
بردار
$\vec V_X$
را بردار شامل همه‌ی متغیر‌های درونی به غیر از 
$X$
تعریف می‌کنیم.
با استفاده از این تعاریف 
توابع متغیر‌های درونی را به صورت زیر تعریف می‌کنیم:
$$
    \f{C_{e,e'}} = \begin{cases}
        true  & \text{ if } e \# e'\\
        false & \text{ otherwise }
    \end{cases}
$$
$$
    \f{M_{s,e}} = \begin{cases}
        Min(s,e) \wedge Con(s) & \text{ if } s \vdash_{min} e \\
        false                  & \text{ otherwise }
    \end{cases}
$$
\begin{align*}
    \f{EN_{s,e}} & =
    \left(
    M_{s,e} \vee
    \left(
    \bigvee_{s'\prec s}EN_{s',e}
    \right)
    \right)
    \bigwedge
    Con(s)
\end{align*}
که در آن‌ها داریم:
\begin{align*}
    Con(s)   & =   \left(
    \bigwedge_{ 1\leq j<j' \leq n \wedge e_j,e_{j'} \in s}
    \neg C_{e_j,e_{j'}}
    \right)               \\
    Min(s,e) & = \left(
    \bigwedge_{s' \subseteq E. (s' \subset s \vee s \subset s')
        \wedge e \notin s'}
    \neg M_{s',e}
    \right)
\end{align*}
فرض کنید که
$\mathbb{E}$
مجموعه‌ی تمامی سه‌تایی‌ها به فرم
$(E,\#',\vdash')$
باشد که داشته باشیم:
\begin{align*}
    \#' \subseteq E \times E \\
    \vdash' \subseteq \mc{P}\times E
\end{align*}
یک تابع به فرم
$ES: \times_{V \in \mathcal{V}\setminus \s{PV}} \mathcal{R}(V) \rightarrow \mathbb{E}$
تعریف می‌کنیم که به صورت شهودی ساختمان رویداد حاصل از مقدار فعلی متغیر‌ها در مدل علی را به دست می‌دهد.
فرض کنیم 
$\vec v$
برداری شامل مقادیر متغیرهای
$\mc{V} \setminus \s{PV}$
باشد.
به ازای هر متغیر مانند 
$V \in \mc{V}$
مقدار آن در 
$\vec v$
را با
$\vec v(V)$
نمایش می‌دهیم.
تابع 
$ES$
را به گونه‌ای تعریف می‌کنیم که اگر 
$ES(\vec v) = (E,\#',\vdash')$
آنگاه داشته باشیم:
\begin{align*}
    \forall e,e' \in E. e \#' e' \wedge e' \#' e
     & \iff \vec{v}(C_{e,e'}) = \T \\
    \forall s \in \mathcal{P}(E), e \in E.  s \vdash' e
     & \iff \vec{v}(EN_{s,e}) = \T
\end{align*}
در نهایت فرض می‌کنیم که رفتار نا‌امن
\lf{Unsafe Behavior}
 سیستمی که در قالب ساختمان رویداد مدل شده است، در قالب تابع متغیر 
$PV$
توصیف شده است و در صورتی که رفتار نا‌امن در سیستم وجود داشته باشد مقدار آن درست و در غیر این صورت غلط است.
با استفاده از مدل علی که به این شکل توصیف شود برای پیدا کردن علت خطا کافی است علت واقعی 
$PV=\T$
را در مدل علی و مطابق تعریف علت واقعی پیدا کنیم.

