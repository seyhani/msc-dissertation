\section{پیدا کردن علت خطا در نت‌کت پویا}

با استفاده از تعاریف بخش‌های قبلی در این بخش به چگونگی پیدا کردن علت خطا در یک برنامه توصیف شده در نت‌کت پویا می‌پردازیم.

فرض می‌کنیم که یک عبارت نت‌کت پویا
$p$
در اختیار داریم.
ابتدا عبارت
$p$
را به فرم نرمال در می‌آوریم.
فرض کنیم عبارت معادل 
$p$
در فرم نرمال
عبارت 
$q$
باشد.
اکنون فرض کنیم 
$\mr{E} = \sem{q}$
ساختمان رویداد معادل 
$q$
باشد.
اکنون مدل علّی 
$\mc{M}$
را بر اساس
$\mr{E}$
می‌سازیم و رفتار ناامن را در قالب تابع متغیر
$PV$
این مدل و به شکل یک شرط بر روی مجموعه‌ی پیکربندی‌های مدل علّی توصیف می‌کنیم.
در نهایت کافی است برای پیدا کردن علت واقعی رفتار ناامن، علت واقعی 
$PV = \T$
در 
$\mc{M}$
را بر اساس تعریف 
\ref{def:extended}
پیدا کنیم.
