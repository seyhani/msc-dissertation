\section{پیدا کردن علت خطا در نت‌کت پویا}

با استفاده از تعاریف بخش‌های قبلی در این بخش به چگونگی پیدا کردن علت خطا در یک برنامه توصیف شده در نت‌کت پویا می‌پردازیم.

فرض می‌کنیم که یک عبارت نت‌کت پویا
$p$
در اختیار داریم.
ابتدا عبارت
$p$
را به فرم نرمال مطابق تعریف 
\ref{def:dynetkat-normal}
در می‌آوریم.
فرض کنیم عبارت 
$q$
فرم نرمال
عبارت 
$p$
باشد.
اکنون فرض کنیم 
$\mr{E} = \sem{q}$
ساختمان رویداد معادل 
$q$
باشد.
اکنون مدل علّی 
$\mc{M}$
را بر اساس
$\mr{E}$
می‌سازیم و رفتار نا امن را در قالب تابع متغیر
$PV$
این مدل و به شکل یک شرط بر روی مجموعه‌ی پیکربندی‌های مدل علّی توصیف می‌کنیم.
در نهایت کافی است برای پیدا کردن علت واقعی رفتار نا امن، علت واقعی 
$PV = \T$
در 
$\mc{M}$
را بر اساس تعریف 
\ref{def:extended}
پیدا کنیم.
توجه کنید که در اینجا محدودیتی برای چگونگی تعریف رفتار نا امن وجود ندارد و یان تعریف می‌تواند هر شرطی بر روی مجموعه‌ی پیکر‌بندی‌های مدل علّی باشد.

مثلا اگر مجموعه‌ای از مثال‌های نقض سیستم در قالب پیکربندی‌های ساختمان رویداد وجود داشته باشد
(مثلا مجموعه‌ی
C
)،
می‌توان رفتار نا امن را وجود یکی از این پیکربندی‌ها در سیستم توصیف کرد:
\begin{align*}
    \f{PV} & = \bigvee_{\forall c \in C} c \in \mc{F}(ES(\vec v))
\end{align*}

در مثالی دیگر اگر رفتار نا امن در قالب یک شرط 
unsafe
روی برچسب‌های سیستم توصیف شده باشد می‌توان رفتار نا امن را وجود یک پیکربندی که شامل یک رویداد که شرط 
unsafe
را برآورده می‌کند توصیف کرد:
\begin{align*}
    \f{PV} & = \exists c \in \mc{F}(ES(\vec v)).\exists e \in c . \text{unsafe}(l(e))
\end{align*}



