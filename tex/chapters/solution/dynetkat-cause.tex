\section{پیدا کردن علت خطا در نت‌کت پویا}

با استفاده از تعاریف بخش‌های قبلی در این بخش به چگونگی پیدا کردن علت خطا در یک برنامه توصیف شده در نت‌کت پویا می‌پردازیم.

فرض می‌کنیم که یک عبارت نت‌کت پویا
$p$
در اختیار داریم.
ابتدا عبارت
$p$
را به فرم نرمال مطابق تعریف 
\ref{def:dynetkat-normal}
در می‌آوریم.
فرض کنیم عبارت 
$q$
فرم نرمال
عبارت 
$p$
باشد.
اکنون فرض کنیم 
$\mr{E} = \sem{q}$
ساختمان رویداد معادل 
$q$
باشد.
اکنون مدل علّی 
$\mc{M}$
را بر اساس
$\mr{E}$
می‌سازیم و رفتار نا امن را در قالب تابع متغیر
$PV$
این مدل و به شکل یک شرط بر روی مجموعه‌ی پیکربندی‌های مدل علّی توصیف می‌کنیم.
در نهایت کافی است برای پیدا کردن علت واقعی رفتار نا امن، علت واقعی 
$PV = \T$
در 
$\mc{M}$
را بر اساس تعریف 
\ref{def:extended}
پیدا کنیم.
توجه کنید که در اینجا محدودیتی برای چگونگی تعریف رفتار نا امن وجود ندارد و این تعریف می‌تواند هر شرطی بر روی مجموعه‌ی پیکر‌بندی‌های مدل علّی باشد.

در این پژوهش دو روش برای توصیف رفتار نا امن مورد استفاده قرار می‌گیرد.
در روش اول رفتار نا امن را به شکل مجموعه‌ای از پیکر‌بندی‌های
نا امن توصیف می‌کنیم.
اگر مجموعه‌ی 
C
شامل پیکربندی‌هایی از سیستم باشد که رفتار نا امن دارند
رفتار نا امن سیستم را می‌توان در قالب تابع زیر تعریف کرد:
\begin{equation}
    \label{eq:unsafe}
    \f{PV} = \bigvee_{c \in C} c \in \mc{F}(ES(\vec v))
\end{equation}

در روش دوم رفتار نا امن را وجود یک پیکربندی شامل رویدادی 
با برچسب نا امن در نظر می‌گیریم.
برای این منظور فرض می‌کنیم که 
$U \subseteq L$
مجموعه‌ی برچسب‌های نا امن سیستم باشد که 
$L$
مجموعه‌ی تمامی برچسب‌های ممکن است و رفتار نا امن را در قالب تابع
زیر توصیف می‌کنیم:
\begin{align*}
    \f{PV} & = \exists c \in \mc{F}(ES(\vec v)).\exists e \in c.
    l(e) \in U
\end{align*}



