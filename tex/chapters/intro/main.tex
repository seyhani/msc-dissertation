% !TeX root=../main.tex
\chapter{مقدمه}
در این فصل به صورت اجمالی به موضوع و اهداف مورد انتظار این پژوهش پرداخته می‌شود و ساختار و محتوای فصول این نوشتار توضیح داده می‌شود.
\section{موضوع پژوهش}
شبکه‌های کامپیوتری یکی از مهم‌ترین اجزای زیرساخت سیستم‌های کامپیوتری هستند
\cite{foerster2018survey}.
شبکه‌های مبتنی بر نرم‌افزار
\lf{Software Defined Network}
با متمرکز کردن رفتار کنترل‌کننده‌ی شبکه و ساده‌تر کردن عناصر شبکه در سطح داده
\lf{Data Plane}
تست و درستی‌سنجی شبکه‌ها را تسهیل کرده‌اند.
با این وجود به دلیل اینکه هر شبکه‌‌ی کامپیوتری ذاتا یک سیستم توزیع‌شده و  ناهمگام
\lf{Asynchronous}
است
اطمینان از درستی شبکه‌های مبتنی بر نرم‌افزار همچنان فرآیندی پیچیده و سخت است.
اما این همه‌ی ماجرا نیست.
درستی‌سنجی
\lf{Verification}
و تست یک شبکه‌ی مبتنی بر نرم‌افزار یا به طور کلی یک نرم‌افزار کامپیوتری قدم اول در فرآیند اشکال‌زدایی
\lf{Debug}
آن است.
روش‌های درستی‌سنجی نرم‌افزار در صورتی که سیستم مورد آزمون ویژگی
\lf{Property}
مورد نظر را برآورده نکند یک مثال‌نقض
\lf{Counterexample}
پیدا کرده و کاربر بر می‌گردانند.
به جز این مثال نقض اطلاعات دیگری به کاربر داده نمی‌شود و در نتیجه برای رفع مشکل سیستم کاربر مجبور است تا با روش‌های ابتکاری و با استفاده از دانش و شهود خود در مورد سیستم منشا مشکل را پیدا، آن را برطرف و فرآیند درستی‌سنجی را تکرار کند.
انگیزه اصلی این پژوهش جایگزین کردن بخش انسانی این فرآیند با روش‌های خودکار است.
در واقع در این پژوهش به دنبال پیدا کردن علت واقعی رخ‌دادن خطا در یک شبکه‌ی مبتنی بر نرم‌افزار هستیم.
\begin{figure}
    \centering
    \begin{tikzpicture}[
            node distance={30mm},
            main/.style = {draw, circle, minimum width=10mm},
            s/.style = {->,thick},
            d/.style = {dashed} ]
        \node[main] (b) {$b$};
        \node[main] (a) [above right of=b] {$a$};
        \node[main] (c) [below right of=a] {$c$};
        \node[main] (d) [right of=c] {$d$};
        % \node[main] (e) [left of=b] {$e$};
        % \node[main] (f) [left of=a] {$f$};
        \draw[green,s] (a) -- (b);
        \draw[green,s,d] (a) -- (c);
        \draw[orange,s,d] (c) -- (b);
        \draw[orange,s] (c) -- (d);
        % \draw[red,s] (b) -- (e);
        % \draw[red,s,d] (b) -- (f);
        \draw pic["$\alpha$",
        draw=blue,->,thick,angle eccentricity=1.2,angle radius=1.2cm] {angle=b--a--c} ;
        \draw pic["$\beta$",
        draw=blue,<-,thick,angle eccentricity=1.5,angle radius=1.2cm] {angle=b--c--d} ;
        % \draw pic["$\gamma$",
        % draw=blue,<-,thick,angle eccentricity=1.5,angle radius=1.2cm] {angle=f--b--e} ;
    \end{tikzpicture}
    \caption{ }
    \label{fig:blacklist:consistent}
\end{figure}

به عنوان مثال شبکه‌ی رسم‌شده در شکل
\ref{fig:blacklist:consistent}
را در نظر بگیرید.
در این شبکه امکان انجام دو به‌روز رسانی 
$\alpha$
و
$\beta$
وجود دارد که به ترتیب مسیر‌های سبز و نارنجی پر رنگ‌ را با مسیر‌های خط‌چین جایگزین ‌می‌کنند.
فرض کنید که ویژگی مورد انتظار در این شبکه نرسیدن بسته‌a
\lf{Packet}
ای از 
$a$
به
$d$
باشد.
پس از درستی‌سنجی می‌توان به مثال نقضی برای این ویژگی دست‌یافت که در آن ابتدا 
به روزرسانی 
$\alpha$
انجام می‌شود و سپس یک بسته از 
$a$
به 
$d$
ارسال می‌شود.
این مثال نقض ساده و کوچک است ولی در نگاه اول و بدون بررسی بیشتر کمک چندانی به اشکال‌زدایی این شبکه نمی‌کند.
با بررسی بیشتر این شبکه می‌توان دریافت که انجام شدن به روز رسانی
$\alpha$
پیش از تکمیل شدن به روز رسانی
$\beta$
سبب به وجود آمدن مسیر بین 
$a$
و
$d$
و در نتیجه بروز خطا می‌شود.
هدف از انجام این پژوهش پیدا کردن چنین عواملی 
(مثلا در اینجا ترتیب رخ‌دادن به‌روز رسانی‌ها)
به عنوان علت واقعی رخ دادن خطا است.

\section{اهداف پژوهش}
شبکه‌های مبتنی بر نرم‌افزار به دلیل اینکه ذاتا توزیع‌شده و ناهمگام هستند مورد استفاده‌ی خوبی برای پیدا کردن علت واقعی خطا و تسهیل فرآیند رفع اشکال هستند. 
به همین دلیل در این پژوهش این دامنه از مسائل مورد بررسی قرار گرفته‌اند.
برای توصیف این شبکه‌ها از زبان نت‌کت پویا
\lf{DyNetKAT} \cite{dynetkat}
استفاده شده است که یک زبان مینیمال و ساده بر پایه‌ی زبان نت‌کت
\lf{NetKAT} \cite{netkat}
که توصیف به‌روز‌ رسانی‌های شبکه و استدلال در مورد چندین بسته موجود در شبکه را فراهم کرده است.
هدف اصلی این پژوهش ارائه‌ی یک فرمولاسیون از علت واقعی
\lf{Actual Cause}
بر اساس تعریف ارائه شده در
\cite{hp}
برای برنامه‌های توصیف شده در زبان نت‌کت پویا و بررسی کارایی علت‌های واقعی پیدا شده در فرآیند اشکال‌زدایی از شبکه است.

\section{ساختار فصل‌ها}
در فصل دوم تعاریف و دانش پیش‌زمینه‌ی مورد نیاز برای بقیه فصول بیان می‌شود.
فصل سوم روش ترجمه‌ی یک برنامه‌ی توصیف شده در زبان نت‌کت پویا به یک مدل علّی
\lf{Causal Model}
بیان می‌شود.
فصل چهارم شامل به کار گیری روش ارائه شده در این پژوهش برای پیدا کردن علت خطا در چند دسته از ویژگی‌های رایج در شبکه است. در این فصل بررسی می‌شود که علت واقعی پیدا شده تا چه میزان با شهود موجود از مساله تطابق دارد و این فرمولاسیون تا چه حد موفق عمل می‌کند.
فصل پنج شامل جمع‌بندی کار‌های انجام شده در این پژوهش و بحث در مورد کاستی‌های آن و کار‌های پیش‌رو است.
در نهایت در فصل ششم مروری بر کار‌های پیشین و مرتبط با این پژوهش انجام شده است.