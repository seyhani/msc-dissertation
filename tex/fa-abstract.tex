جایگزینی شبکه‌های کامپیوتری سنتی با شبکه‌ها مبتنی بر نرم‌افزار%
\lf{Software-Defined Network}
باعث شده است تا استفاده از روش‌ها و ابزار‌های مبتنی بر روش‌های صوری%
\lf{Formal Methods}
برای درستی‌سنجی%
\lf{Verification}
این شبکه‌ها تسهیل شود.
با اینکه درستی‌سنجی لازمه‌ی رفع‌ایراد در سیستم‌ها است، اما پس از مشخص شدن وجود خطا در سیستم، پیدا کردن دلیل این مساله که چرا سیستم دچار خطا شده است همچنان بر عهده‌ی کاربر است و لازم است که او با تحلیل و بررسی سیستم علت مشکل را پیدا کند و ایراد سیستم را بر طرف کند.
ابزار‌های درستی‌سنجی در صورتی که سیستم مطابق انتظار رفتار نکند یک مثال‌نقض یا گواهی برای اثبات این موضوع به کاربر ارائه می‌کنند اما این مدارک به تنهایی و بدون پردازش بیشتر اطلاعات کافی از چرایی مشکل در اختیار نمی‌گذارند.

توضیح پدیده‌ها در متون فلسفه قرن‌ها مورد مطالعه قرار گرفته است و نتایج این مطالعات در قالب اصول استدلال مبتنی بر خلاف واقع%
\lf{Counterfactual Reasoning}
تجمیع شده است.
هالپرن%
\lf{Joseph Y. Halpern}
و پرل%
\lf{Judea Pearl}
یک فرمولاسیون ریاضی برای پیدا کردن علت واقعی%
\lf{Actual Cause}
بر اساس استدلال مبتنی بر خلاف واقع ارائه کرده‌اند که در پژوهش‌های متعددی برای ارائه توضیح در مورد خطای رخ داده در سیستم و پیدا کردن علت واقعی آن مورد استفاده قرار گرفته است.

در این پژوهش از مفهوم علت واقعی ارائه شده توسط هالپرن و پرل برای توضیح خطا در شبکه‌های مبتنی بر نرم‌افزار استفاده می‌شود.
به صورت دقیق‌تر در این پژوهش یک مدل علّی%
\lf{Casual Model}
ارائه می‌شود که با استفاده از آن می‌توان در مورد ساختارهای شبکه، مانند وجود هم‌روندی یا ترتیب میان به‌روز رسانی‌های شبکه، برای پیدا کردن علت واقعی رفتار نا امن شبکه استدلال کرد.

در این پژوهش برای توصیف نرم‌افزار شبکه از زبان نت‌کت‌ پویا%
\lf{DyNetKAT}
استفاده می‌شود.
نت‌کت پویا یک سطح‌ بالا برنامه‌ نویسی شبکه است که بر پایه نت‌کت%
\lf{NetKAT}
 بنا شده و با وجود اینکه مینیمال بودن آن را حفظ کرده امکان توصیف به‌روز رسانی‌های پویای شبکه را فراهم می‌کند.
در این پژوهش از ساختمان رویداد%
\lf{Event Structure}
به عنوان مدل معنایی%
\lf{Semantic Model}
برنامه‌های نت‌کت پویا استفاده می‌شود.
ساختمان رویداد امکان توصیف صریح هم‌روندی را فراهم می‌کند که این موضوع سبب می‌شود چنین روابطی میان پردازه‌ها را هم بتوان به عنوان علت خطا در نظر گرفت، امری که با استفاده از مدل‌های برگ‌برگ شده%
\lf{Interleaving}
امکان پذیر نیست.
روش ارائه شده در این پژوهش برای پیدا کردن علت نقض چند ویژگی مطرح 
شبکه، مثلا نبود دور، مورد استفاده قرار گرفته است و بررسی این مساله‌ها نشان می‌دهد که علت‌های پیدا شده با شهود موجود از علت خطا در سیستم تطابق دارند.
