جایگزینی شبکه‌های کامپیوتری سنتی با شبکه‌ها مبتنی بر نرم‌افزار باعث شده است تا استفاده از روش‌ها و ابزار‌های مبتنی بر روش‌های صوری برای درستی‌سنجی این شبکه‌ها تسهیل شود.
با اینکه درستی‌سنجی لازمه‌ی رفع‌ایراد در سیستم‌ها است، اما پس از مشخص شدن وجود خطا در سیستم، پیدا کردن دلیل این مساله که چرا سیستم دچار خطا شده است همچنان بر عهده‌ی کاربر است و لازم است که او با تحلیل و بررسی سیستم علت مشکل را پیدا کند و ایراد سیستم را بر طرف کند.
ابزار‌های درستی‌سنجی در صورتی که سیستم مطابق انتظار رفتار نکند یک مثال‌نقض یا گواهی برای اثبات این موضوع به کاربر بر می‌گردانند اما این مدارک به تنهایی و بدون پردازش بیشتر اطلاعات کافی از چرایی مشکل را در اختیار نمی‌گذارند.

توضیح پدیده‌ها در متون فلسفه قرن‌ها مورد مطالعه قرار گرفته است و نتایج این مطالعات در قالب اصول استدلال مبتنی بر خلاف واقع تجمیع شده است.
هالپرن و پرل یک فرمولاسیون ریاضی برای پیدا کردن علت واقعی بر اساس استدلال مبتنی بر خلاف واقع ارائه کرده‌اند که در پژوهش‌های متعددی برای ارائه توضیح در مورد خطای رخ داده در سیستم و پیدا کردن علت واقعی آن مورد استفاده قرار گرفته است.

در این پژوهش از مفهوم علت واقعی ارائه شده توسط هالپرن و پرل برای توضیح خطا در شبکه‌های مبتنی بر نرم‌افزار استفاده می‌شود.
به صورت دقیق‌تر در این پژوهش یک مدل علّی ارائه می‌شود که با استفاده از آن می‌توان در مورد ساختارهای شبکه، مانند وجود هم‌روندی یا ترتیب میان به‌روز رسانی‌های شبکه، برای پیدا کردن علت واقعی رفتار نا امن شبکه که منجر نقض ویژگی‌های توصیف شده برای آن می‌شود استدلال کرد.

برای توصیف نرم‌افزار شبکه از زبان نت‌کت‌ پویا استفاده می‌شود.
نت‌کت پویا یک زبان ساده و سطح‌ بالا برنامه‌ نویسی شبکه است که امکان ترجمه به کد سوییچ‌های 
OpenFlow
دارد.
نت‌کت پویا بر پایه نت‌کت بنا شده است و با وجود اینکه مینیمال بودن آن را حفظ کرده امکان توصیف به‌روز رسانی‌های پویای شبکه را فراهم می‌کند.
در این پژوهش از ساختمان رویداد به عنوان مدل معنایی برنامه‌های نت‌کت پویا استفاده می‌شود.
ساختمان رویداد امکان توصیف صریح هم‌روندی را فراهم می‌کند که این موضوع سبب می‌شود چنین روابطی میان پردازه‌ها را هم بتوان به عنوان علت خطا در نظر گرفت، امری که با استفاده از مدل‌های برگ‌برگ شده امکان پذیر نیست.
از مدل ارائه شده در این پژوهش برای پیدا کردن علت نقض برخی از ویژگی‌های مطرح شبکه استفاده شده است.
با وجود اینکه معیار مشخصی برای بررسی کیفیت مدل و تحلیل علّی انجام شده وجود ندارد، علت‌های پیدا شده توسط مدل پیشنهادی این پژوهش برای ویژگی‌های بررسی شده با شهود موجود از مساله تطابق دارد.