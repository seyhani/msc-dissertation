The emerging use of Software-Defined Networks instead of 
traditional networking approaches have enabled us to 
use formal verification methods more frequently on such 
systems which result in more reliable networks.
But when an error is detected, the task of finding why  
the system doesn't work as expected remains as a burden 
in front of debugging procedure and the output of 
verification tools such as certificates or counterexamples
does not provide enough understanding of the problem.
Counterfactual causal reasoning, which has a long research
history in the philosophy literature is concerned with 
explaining phenomena by finding their cause.
Halpern and Pearl provided a mathematical formulation of 
the actual cause based on Counterfactual reasoning.
This formulation has been used in several researches in 
order to use the actual cause of an error to provide 
an explanation of the system problem.

In this research, we use this notion of causality in the 
context of Software-Defined Networks. 
More specifically we provide a causal model which can be used
to reason about which constructs of the network, such as 
existence of concurrency or order between network updates, can
be considered as an actual cause of the network's unsafe behavior
which results in a property violation.

To encode network programs we use DyNetKAT, which is a simple and high level network programming language that can be compiled into
OpenFlow switches flow table. DyNetKAT is build upon NetKAT and
while preserves its minimality, it enables the encoding of 
dynamic updates in networks. We use event structures as a semantic
model of DyNetKAT programs which allows us to explicitly encode 
concurrency, so such relations may also be considered as a cause
while this is not possible in an interleaving model. 
We used our model to explain the violation of some well-known 
network properties and while there is no measurement or method to 
evaluate causal analysis, it seems that causes of our model
matches our intuition of the root of the problem in the network.

