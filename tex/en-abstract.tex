The emerging use of Software-Defined Networks instead of 
traditional networking approaches have enabled us to 
use formal verification methods more easily on such 
systems which result in more reliable networks.
But when an error is detected, the task of finding why  
the system doesn't work as expected still remains and a person
needs to manually investigate the problem and find out how 
to fix the system. 
In the case of failure, verification tools produce 
a certificate or counterexample but such evidence does not
provide enough understanding of the problem.

Explaining phenomena with causal reasoning has been studied for
centuries in the philosophical literature which is distilled as
the counterfactuality principle.
Halpern and Pearl proposed a mathematical formulation of 
the actual cause based on Counterfactual reasoning.
This formulation has been used in several research projects in 
order to use the actual cause of an error to provide 
an explanation of the system's problem.

In this research, we use Halpern and Pearl's notion of causality 
in the context of Software-Defined Networks to explain failures.
More specifically we provide a causal model which can be used
to reason about which constructs of the network, such as 
existence of concurrency or order between network updates, can
be considered as an actual cause of the network's unsafe behavior
which results in a property violation.

To encode network programs we use DyNetKAT, which is a simple and 
high-level network programming language that can be compiled into
the flow table of OpenFlow switches. DyNetKAT is built upon 
NetKAT and while preserving its minimality, it enables the
 encoding of dynamic updates in networks. We use event structures
as a semantic
model of DyNetKAT programs which allows us to explicitly encode 
concurrency and as a result, such relations may also be 
considered as a cause while this is not possible in an 
interleaving model. 
We used our model to explain the violation of some well-known 
network properties by finding the actual cause of the failure.
While there is no measurement or method to 
evaluate causal analysis, it seems that the causes found by our 
model matches the intuition of the root cause of the problem in 
the network.